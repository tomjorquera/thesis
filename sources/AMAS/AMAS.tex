\part{Multi-Agents Systems and AMAS Theory}

As we said at the end of the last part, providing a method able to scale to the needs of the full range of optimization problems would require to it to be capable of adapting itself to the problem at hand.
The main theme of the SMAC team\footnote{\emph{Systèmes Multi-Agents Coopératifs} (Cooperative Multi-Agents Systems)}, in which this thesis has been realized, in the Adaptive Multi-Agent Systems (AMAS) Theory. This theory relates to the design of agent-based complex systems with self-adaptive capabilities.

In this part we will present first a short history of multi-agents systems, before concentrating on the concepts of the AMAS theory.

\section{Multi-Agents Systems}

Multi-Agents Systems (MAS) are a relatively recent field which can be seen as the intersection of Artificial Intelligence (AI) and Systems Theory. As a reminder, the AI field was developed in 1950s as "the science and engineering of making intelligent machines." [[source MacCarthy cf wikipedia]] This rather ambitious project was somewhat toned down during the 70s when the field was the subject of several setbacks leading to an "AI winter"[[REF ? Russell and Norvig ?]], which effects can still be felt today. The commonly accepted reason for this setback was that the researchers have been too much ambitious in their expectations of the breakthroughs which would be produced by the field, and did not take enough in account the inherent complexity of some of the task they were proposing to handle (\emph{e.g. language processing).

This "disgrace" period of the IA field was ended with the success of expert systems in the 80s. These systems aim to emulate the ability of a human being to take decisions based on expert knowledge, using inference mechanisms (via an \emph{inference engine}) and a rule database.

[[TO PUT? Several specific subfield subfields of AI have been defined, among which we can find automated problem solving, machine learning, robotics, knowledge engineering, planning, affective computing \emph{etc.}]]