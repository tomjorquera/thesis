\part{Collective Problem Solving Patterns}

In this part, we take a step back from the MAS we described in the preceding parts.
Numerical optimization was a mostly unexplored application domain in regard to multi-agents based algorithms. by taking the (somewhat ambitious) task to propose a MAS which would be applicable for this domain in its entirety.

As numerical optimization is in itself an abstract mathematical field, we too had to abstract ourselves from concrete applications. We did not have the possibility to reduce the set of possible configurations and thus we had the occasion to encounter a variety of problems which have been mostly ignored before. Indeed, the graph representations of numerical optimization problems are quite diverse, and can present some topological properties not present in classical MAS formalisms. [[DCOP for example (presented in [[REF]]), ]]

In the description of our system, we presented a set of NCS [[use acronym ?]], and the specifics mechanisms we introduced to handle them. We believe that these NCS are only the instantiation of more general problematic configuration, which we dub \emph{Collective Problem Solving Patterns} (CPSP). The patterns are not restricted to numerical optimization, they can potentially be encountered in all sorts of application domains.

Architecture and software development [[beneficed greatly from]] the identification of common design pattern. In the same regard, we believe that the identification of these problem patterns as such, as well as the proposal of solutions to handle them, could lead to a great improvement for the design of agent-based systems for problem solving as a whole.

Consequently, we will present in this part the CPSP which can be derived from the NCS we identified during the design of our system.


\section{Introduction - Collective Problem Solving Patterns are not Design Patterns}

[[EXPLAIN THE DIFFERENCE WITH EXISTING AGENT PATTERNS]]

Before describing the CPSP in themselves, we must explain how these patterns differ from the existing design patterns for MAS.

There is already an existing (if limited) corpus of design patterns for MAS. These patterns have usually in scope either the design of the organizational structure of system or the design of the behavior architecture of the agents. These patterns concerns the \emph{design} of the system regarding the target application domain. These patterns are relevant in the design of the \emph{organization of the system}, according to the application domain.
What we propose here is a different sort of patterns, which concerns the \emph{behavior of the agent}, according to an existing organization. Design Patterns concern the structural aspects of the system, while Problem Solving Patterns concerns its functional aspects.

In this regard, CPSP are less generic than Design Patterns. Indeed the latter can be applied to the whole range of MAS, while the former only concerns MAS designed for problem solving (excluding, for example, MAS for simulation).

These two kinds of patterns should be seen as complementary. First the designer could use design pattern to design the structure of the MAS according to the needs of the application domain. Then he could use CPSP to identify and solve specific problems resulting from such modeling.

\section{Description of a Problem Solving Pattern}

\subsection{Agent Roles}

\subsection{Problem Pattern}

\subsection{Solution}

\section{Identified Collective Problem Solving Patterns}

In this section, we present the CPSP we identified from the NCS we encountered during the design of our MAS.

\subsection{Conflicting Requests and Criticality}

\subsection{Cooperative Trajectories}

\subsection{Cycle Solving}

\subsection{Hidden Dependencies}

\subsection{Asynchronous Requests}

\section{Conclusion on Collective Problem Solving Patterns}
We identified these patterns from the the NCS of our system. In reverse, the NCS used in the design of AMAS seems perfectly adequate to instantiate known CPSP. Indeed, subsumption-based behavior architectures are appropriate the model this kind of "exception"-like situations. Should CPSP become more wildly used, one could expect this way of modeling behavior to become quite popular.