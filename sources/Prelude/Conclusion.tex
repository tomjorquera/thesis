\Conclusion{Conclusion and Perspectives}

In this thesis we identified a severe limitation of current continuous optimization methods regarding the handling of complex continuation problem. Problems of this category are usually too complex to be solved by classical optimization methods because of multiple factors: the interdependencies of their components, their heavy computational cost, their nonlinearities \emph{etc.} This limitation has been the motivation to propose new specific methods which divide the problem into several disciplines and distribute the optimization process using discipline-level optimizers. However these methods are often difficult to put in practice and cumbersome, not suiting the need of a flexible and iterative process often associated with such problems.

\section*{Thesis Contributions to Continuous Optimization}

This thesis proposes a \textbf{new approach for solving complex continuous problems using an adaptive multi-agent system}. This system, designed following the Adaptive Multi-Agent Systems theory, proposes a decentralized way to automatically distribute the optimization process among the agents, and is able not only to solve large-scale complex problems, but also to adapt to changes made by the user during optimization.\\
The scalability of our approach is made possible by the fact that the agents keep a local view of the system. The system adapts to changes by propagating them from neighbor to neighbor, enabling the \textbf{interactive co-design of the solution}.

This system is built upon a \textbf{general continuous problem modeling we named Natural Domain Modeling for Optimization}, which transform the optimization problem into an entities graph. \textbf{This transformation does not require any simplification, modification or reformulation of the original problem and is fully automatic}.

Following the AMAS theory, we kept the agent perceptions and capabilities at a local level, allowing them to communicate and interact only with their immediate neighbors. Doing so, we are able to handle the problems complexity, as each agent keeps a local point-of-view. The agents are able to use external optimization tools (or can alternatively use local approximation techniques) to solve their local optimization problems. \textbf{This local optimization behavior, along with the message-based global consistency, enables a nominal distributed optimization process}.

We identified several configurations that are susceptible to disturb the good functioning of our system optimization process, corresponding to Non Cooperative Situations (NCS) of the AMAS theory. While these NCS can arise from the interaction of several agents, the agents had to be able to detect and solve them using only local mechanisms. \textbf{For each NCS we proposed local cooperative behaviors for the agents to be able able to cooperatively solve the situation and restore the correct optimization flow}. These cooperative behaviors use specific mechanisms and measures in order to correctly identify the NCS and take the adequate corrective action.

We proved the modularity of our design by showing how our system could be modified to handle additional concerns. To illustrate this, we \textbf{integrated in the agent mechanisms for managing the uncertainties propagation}, effectively allowing the system to realize optimization under uncertainties.

At last, we validated our approach on several test cases. We also integrated our system into a prototype in the context of the ID4CS project funded by the French National Research Agency, prototype currently tested by our industrial parters Airbus and Snecma.

\section*{Thesis Contribution to Multi-Agent Systems}

We already introduced how our identification of specific NCS led to the development of specific agent mechanisms in order for the system to maintain a correct behavior, such as cooperative trajectories, cycle solving \emph{etc.} Based on these NCS and solving mechanisms, we proposed more general Collective Problem Solving Patterns (CPSP). \textbf{These CPSP provide some guidance to the MAS designer regarding some potentially problematic agent configurations which may happen in the system, and propose some solving mechanisms to handle these situations.} We illustrated these CPSP with blueprints summarizing the configuration and mechanisms involved.\\
This work contributes to the ongoing effort of creating general tools for the design of AMAS in the context of problem solving, regrouped under the name AMAS4Opt (Adaptive Multi-Agents Systems for Optimization).

Using the Make Agents Yourself framework, we proposed \textbf{a modular agent architecture adapted to the modeling of hierarchical AMAS agent roles}, based on reusable \enquote{building blocks}. This architecture use a composition of stackable \enquote{skills} components, which allows for an efficient implementation of the handling of the different NCS by the agents.\\
This architecture can be used by the MAS designer as a base to design his own agents, using existing components, and contribute to the effort of providing tools for MAS engineering.

We proposed \textbf{a general graph representation of continuous optimization problems, which can be re-used by MAS designers as a common base to propose other MAS-based approaches for continuous optimization}. Using this common representation, such methods will be easier to compare in terms of solving mechanisms and performances.

\section*{Scientific Perspectives}

\subsection*{Perspectives on MAS for Continuous Optimization}

In regard of continuous optimization, our system could be enhanced with several additional capabilities. Currently, our system concentrates on providing \emph{one} optimal solution. An obvious improvement would be to modify the agents to explore the Pareto front and provide several optimal solutions to the problem. A possible lead would be to modify the objective agents handling of criticality, in order to modulate the weight associated with each objective.

Another possible improvement for designers would be to integrate multi-fidelity models in the system. Multi-fidelity model is a technique to reduce the computational cost of optimization, using several versions of the same model with different computational costs. The low-cost, imprecise models are used at the start of the optimization process, while the high-cost, high-fidelity models are used when the system is starting to converge, in order to improve the precision of the solution. A difficulty of such mechanisms is the maintaining of the consistency between the different levels. This kind of mechanisms could be implemented in our system through the behavior of model agents.

The use of external optimizers could be improved by providing automated optimizer selection mechanisms, in order for the agents to be able to select the most appropriate optimizer, or even change of optimization method during the solving process. Such improvement would possibility require the creation of an optimization ontology in order to characterize the different optimizers.

One could imagine to add self-organizing capabilities to the system, in order to automatically compose the agent graph representing an optimization problem. Such functionality could be used, for example, to provide assistance to the designer during the specification of the optimization problem.\\
With such mechanisms, the user would only have to define a base of elements (variables, models, criteria \emph{etc.}) and the system would be able to automatically assemble the elements into an agent graph, possibly creating (or requesting to the user) new elements as needed. Such functioning would be made possible more easily by the fact that the NDMO formalism does not necessitate any specific formulation to be valid.

\subsection*{Perspectives on the Design of MAS and CPSP}

Concerning the design of MAS, we believe that both researchers and engineers would benefit greatly from the creation of an agent patterns repository. We intend to provide a more detailed and standardized description of the Collective Problem Solving Patterns we identified, with the goal of having a self-sufficient specification document.

An obvious continuation of this work is the identification of new CPSP, either concerning configurations we failed to identify in our application, or with configuration which appear in other application domains. Another possibility concerns the development of alternate handling mechanisms for existing CPSP.

An interesting addition to the CPSP would be to provide an implementation of specialized components which only need to be completed by providing specific functions based on the application domain. To which extend such partial instantiation of the CPSP is possible is an open question at this moment.

\subsection*{Perspectives on the AMAS Theory}

At last, we would like to finish by discussing some interesting observations concerning the AMAS theory in itself and the identification of NCS. Prior works using the AMAS theory concentrated on the identification of NCS at a given moment. The detection of the NCS was usually done immediately by the agents, and the corrective actions were relatively simple and direct.

Most of the NCS we identified were quite different in their functioning. These NCS not only require the agents to cooperate over several iterations to be solved, but their identification itself requires the agents to take additional measures. Moreover, some of the configurations we identified are not systematically problematic in themselves, but only \emph{potentially} problematic depending on some of the parameters (naturally converging cycles, hidden dependencies with adequate influences \emph{etc.}). The common point among these situations is how they are related to the \emph{dynamics} of the system, about its evolution toward one direction or another.

This observation leads to the question of whether a possible distinction could be made between \emph{spatial} NCS, corresponding to the interactions between agents at a given instant, and \emph{temporal} NCS, corresponding to the evolution of an agent over time. If such distinction proved to be relevant, it could lead to new insights on the design of AMAS-based systems, and possibly on the AMAS theory in itself.

