\Conclusion{Conclusion and Perspectives}

In this thesis we identified a severe limitation of current continuous optimization methods regarding the handling of complex continuation problem. These problems are usually  too complex to be solved by classical optimization methods, because of multiple factors: the interdependencies of their components, their heavy computational cost, their nonlinearities \emph{etc.}. Because of this reason, specific methods have been developed to distribute the optimization process. However these methods are often difficult to put in practice and cumbersome, not suiting the need of a flexible and iterative process often associated with such problems.

We presented in this work a \emph{new approach for solving complex continuous problems using an adaptive multi-agent system}. This system, designed following the AMAS theory, proposes a decentralized way to automatically distribute the optimization process among the agents. This system is build upon a \emph{general continuous problem modeling we named NDMO}, which transform the optimization problem into an entities graph. This transformation is, once more, fully automatic, and does not require any simplification, modification or reformulation of the original problem. While our system is the first instance of MAS using this modeling, this graph representation is generic and can be re-used as a base to propose other comparable systems.\\
Following the AMAS theory, we kept the agent perceptions and capabilities at a local level, allowing them to communicate and interact only with their immediate neighbors. Doing so, we are able to handle the problems complexity, as each agent keeps a local point-of-view. In order to maintain a global consistency, the agents exchange inform and request messages that are propagated into the system.

