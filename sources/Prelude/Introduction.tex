\Introduction{Introduction} \label{introduction}

\section*{Complex Continuous Optimization and Multi-Agent Systems}

Continuous optimization is a very large field including various methods tailored for diverse specific requirements. While this approach was successful in providing a toolbox of specialized methods, the evolution of industrial needs draws attention to some of its limitations. Indeed, current optimization methods fail to handle the more complex optimization problems. These problem are characterized by heavy calculus, the many interdependencies between their components and the diverse expertise domains they involve. Classical optimization tools struggle with these problems because of these factors, and specific methods have been proposed to handle this complexity, giving birth to the field of Multidisciplinary Design Optimization (MDO). However,  MDO methods involve possibly important transformations to the original problem in order to divide the problem into simpler ones, which make this approach somewhat cumbersome and potentially inefficient for highly connected problems.

At the same time, new paradigms are being proposed to handle systemic complexity. One of the most successful is the field of Multi-Agent Systems (MAS). This approach proposes to handle problem complexity using systems of interconnected agents. Instead of reducing the problem in order to solve it using a centralized process, MAS techniques preserve the original problem and use decentralized mechanisms in order to spread the solving effort among the agents. MAS has proved successful in the field of combinatorial optimization, on problems such as graph coloring, sensors network or scheduling.

During the last ten years, the scientific community has relentlessly pursued the effort to bridge the gap between these two apparently irreconcilable approaches: mathematical optimization and MAS. The goal of this thesis is to contribute to this effort by addressing this mostly unexplored potential application field of MAS: complex continuous optimization.

\section*{Contributions of the Thesis}

The main contribution of this thesis concerns the applicability of MAS for continuous optimization We study continuous optimization problem and show how all of them share a common structure. Using this observation, we propose a representation of continuous optimization problems entities graphs, which we call Natural Domain Modeling for Optimization (NDMO). Based on this representation we identify several agent roles for the graph entities. For each agent role we propose a nominal behavior in order to produce a MAS capable of distributing the optimization process. In accordance with the AMAS theory, we identify a set of Non-Cooperative Situations (NCSs) susceptible to disturb the normal optimization process, and propose a set of cooperation mechanisms to handles them. We demonstrate the modularity of our system by introducing additional concerns with the handling of uncertainties propagations.

This thesis also provides two smaller contributions concerning the deisgn of MAS.  First of all, using the Make Agent Yourself framework  we propose a component-based architecture for AMAS adapted to the handling of multiple agent roles and NCS-related mechanisms. This architecture is based on the idea of stackable skills components following the hierarchy of agent roles, providing the correct methods at the required level.

We also provide a more theoretical contribution by abstracting the NCS and solving mechanisms into more general Collective Problem Solving Patterns (CPSP). These CPSP are based on a more high-level agent role representation, and are abstracted from any direct application domain. They represent specific agent topologies which can be encountered in agent organizations leading to a disruption of the correct system function, as well as of solving mechanisms proposed to handle such configurations. We propose a schematic \enquote{blueprint} representation which synthesize the content of the different patterns.

\section*{Manuscript Organisation}
This thesis is divided into 4 parts:
\begin{enumerate}[P{a}rt I.] %the {a} avoids this letter to be used as the counter (the I is used instead)
\item This part introduces the context of the study by presenting an overview of the continuous optimization field, MAS for optimization and the Adaptive Multi-Agent Systems theory.
\item This part presents the contribution of this thesis: a MAS for solving continuous optimization problems. We propose a modeling of a continuous optimization problem as an agent graph, and describe some cooperative behaviors for the different agent roles.
\item [[TODO: depends on CPSP chapter]].
\item In this part we present the experiments we did in order to evaluate and validate our approach.
\end{enumerate}