\if@doubleinterligne\renewcommand\baselinestretch{0.5}\fi
  \thispagestyle{plain}
  %\vspace*{2cm} 
  %
  %POUR DECALER VERS LE BAS LA PAGE DE TITRE%
  \begin{center}
    \Large Tom Jorquera\\
    \addvspace{2mm}
    \textsc{\textbf{An Adaptive Multi-Agent System for Self-Organizing Continuous Optimization}}\\
	\addvspace{2mm}
    \begin{center}
   	 \large\normalfont Supervisors: Marie-Pierre \textsc{Gleizes}, Jean-Pierre \textsc{Georgé}\\
   	  \large\normalfont \emph{Université de Toulouse III}\\
     \end{center}
     \\
    \addvspace{3mm}
    \textbf{\hrulefill~ Abstract ~\hrulefill}\\  
     
  \end{center}
  \vspace{-0.3cm}
        
<\normalsize \hspace{0.6cm}Cette thèse présente une nouvelle approche pour la distribution de processus d’optimisation continue dans un réseau d'agents coopératifs. L’optimisation continue est un très large champ de recherche, incluant de multiples sous-domaines visant à résoudre efficacement un sous-ensemble spécifique de problèmes. Bien que cette approche ait eu un fort succès dans de nombreux domaines d'applications, elle a montré ses limites sur des problèmes d'optimisation hautement complexes, tels que l’optimisation de conception de systèmes complexes. Ce type de problèmes implique généralement un grand nombre de modèles hétérogènes provenant de plusieurs disciplines interdépendantes.

Dans le but de résoudre de tels problèmes, le domaine de l’optimisation multidisciplinaire a été proposé. Les méthodes d'optimisation multidisciplinaire proposent de distribuer le processus d'optimisation, généralement en reformulant le problème original d'une manière qui réduit les interconnexions entre les disciplines. Cependant, ces méthodes présentent des désavantages en ce qui concerne la difficulté de les appliquer correctement, ainsi que leur manque de flexibilité.

En utilisant la théorie des AMAS (Adaptive Multi-Agent Systems), nous proposons un système multi-agent qui distribue le processus d'optimisation, appliquant des optimisations locales aux différentes parties du problème, tout en maintenant un état global consistent. La théorie des AMAS, développée par l'équipe SMAC, se concentre sur la coopération comme mécanisme fondamental pour la conception de systèmes complexes artificiels. Cette théorie prône la conception d'agents auto-adaptatifs, interagissant de manière coopérative au niveau local afin de s'aider mutuellement à atteindre leurs buts.

En se basant sur la théorie des AMAS, nous proposent une représentation générique à base d'agents des problèmes d'optimisation continue. A partir de cette représentation, nous proposons un comportement nominal pour les agents afin d'exécuter le processus d'optimisation. Nous identifions ensuite certaines configurations spécifiques qui pourraient perturber le processus, et présentons un ensemble de comportements coopératifs pour les agents afin d'identifier et de résoudre ces configurations problématiques.

Enfin, nous utilisons les mécanismes de coopération que nous avons introduit comme base à des patterns de résolution coopérative de problèmes. Ces patterns sont des recommandations de haut niveau pour identifier et résoudre des configurations potentiellement problématiques qui peuvent survenir au sein de systèmes de résolution collective de problèmes. Ils fournissent chacun un mécanisme de résolution coopérative pour les agents, en utilisant des indicateurs abstraits qui doivent être instanciés pour le problème en cours.

Nous validons notre système sur plusieurs cas tests, en utilisant des problèmes d'optimisation bien connus ainsi que des benchmarks d'optimisation multidisciplinaire. Afin d'étudier les propriétés de passage à l'échelle de notre système, nous proposons deux manières de générer des problèmes valides d'optimisation. En utilisant ces techniques nous générons des ensembles de cas tests conséquents qui nous permettent de valider plusieurs propriétés de notre système.

\vspace{-0.4cm}
\begin{center}{\hrulefill}\end{center}
\cleardoublepage